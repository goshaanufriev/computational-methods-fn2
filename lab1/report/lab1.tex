% !TeX spellcheck = ru_RU-Russian
\documentclass[12pt, a4paper]{article}

\usepackage[utf8]{inputenc}
\usepackage[T1]{fontenc}
\usepackage[russian]{babel}

\usepackage[oglav, boldsect, eqwhole, figwhole, %
remarks, hyperref, hyperprint]{fn2kursstyle}

\makeatletter
\renewcommand{\@oddhead}{\vbox{Ануфриев Г.А., Брегадзе З.Г. \hfill ФН2-52Б \hrule}}
\makeatother

\begin{document}
	\section*{Ответы на контрольные вопросы}
	\begin{enumerate}
		\item \textbf{Каковы условия применимости метода Гаусса без выбора и с выбором ведущего элемента?}
		
		Пусть СЛАУ задана в матричном виде как $Ax=b$. Тогда главным условием применимости метода Гаусса в обоих упомянутых в вопросе случаях является неравенство нулю определителя матрицы системы: $\det A\neq0$.
		
		Если метод Гаусса применяется без выбора ведущего элемента, то необходимо учитывать следующее условие: $a_{ii}^{(i-1)}\neq0$. Это нужно для того, чтобы избежать деления на ноль и, как следствие, аварийного завершения программы.
		
		В случае выбора главного элемента достаточно неравенства нулю определителя матрицы системы, так как деление производится на наибольший по модулю коэффициент при $a_{ii}$, что обеспечивает устойчивость вычислений.
	\end{enumerate}
	
\end{document}