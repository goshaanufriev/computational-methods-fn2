% !TeX spellcheck = ru_RU-Russian
\documentclass[12pt, a4paper]{article}

\usepackage[utf8]{inputenc}
\usepackage[T1]{fontenc}
\usepackage[russian]{babel}

\usepackage[oglav, boldsect, eqwhole, figwhole, %
remarks, hyperref, hyperprint]{fn2kursstyle}

\makeatletter
\renewcommand{\@oddhead}{\vbox{Ануфриев Г.А., Брегадзе З.Г. \hfill ФН2-52Б \hrule}}
\makeatother

\begin{document}
	\section*{Ответы на контрольные вопросы}
	\begin{enumerate}
		\item \textbf{Каковы условия применимости метода Гаусса без выбора и с выбором ведущего элемента?}
		
		Пусть СЛАУ задана в матричном виде как $Ax=b$. Тогда главным условием применимости метода Гаусса в обоих упомянутых в вопросе случаях является неравенство нулю определителя матрицы системы: $\det A\neq0$.
		
		Если метод Гаусса применяется без выбора ведущего элемента, то необходимо учитывать следующее условие: $a_{ii}^{(i-1)}\neq0$. Это нужно для того, чтобы избежать деления на ноль и, как следствие, аварийного завершения программы.
		
		В случае выбора главного элемента достаточно неравенства нулю определителя матрицы системы, так как деление производится на наибольший по модулю коэффициент при $a_{ii}$, что обеспечивает устойчивость вычислений.
		
		\item \textbf{Докажите, что если $\det A \neq 0$ , то при выборе главного элемента в столбце среди элементов, лежащих не выше главной диагонали, всегда найдется хотя бы один элемент, отличный от нуля.}
		
		$\Box$ <<От противного>>:
		
		Пусть на $k$-том шаге метода Гаусса все элементы в $k$-том столбце начиная с $k$-того равны нулю. Тогда $k$-тый столбец является линейной комбинацией первых $k-1$ столбцов $\Rightarrow$ определитель матрицы равен нулю. Так как элементарные преобразования не обнуляют определитель, а для ихсодной матрицы $\det A \neq 0$, получим противоречие $\Box$
		
		\item\textbf{В методе Гаусса с полным выбором ведущего элемента
		приходится не только переставлять уравнения, но и менять нумерацию неизвестных. Предложите алгоритм, позволяющий восстановить первоначальный порядок неизвестных.}
		
		Для метода Гаусса с полным выбором ведущего элемента необходимо завести массив перестановок, который будет учитывать порядок переменных в ответе. То есть изначально он может выглядеть так:
		
		$$
		\text{permutations} = [0,1,\dots,n-1].
		$$
		
		Если в ходе работы программы меняются местами $i$-й и $j$-й столбцы, то меняются местами и числа $i$ и $j$ в массиве перестановок. Затем, когда найден вектор решения $\tilde{X}$ с измененным порядком неизвестных, находим $X$ с правильным порядком переменных.
		
		\item \textbf{Оцените количество арифметических операций, требуемых для $QR$-разложения произвольной матрицы $A$ размера $n\times n$.}
		
		\item \textbf{Что такое число обусловленности и что оно характеризует? Имеется ли связь между обусловленностью и величиной определителя матрицы? Как влияет выбор нормы матрицы на оценку числа обусловленности?}
		
		Величину
		$$
		\text{cond} A = \|A^{-1}\|\cdot\|A\|
		$$
		называют числом обусловленности матрицы $A$. Матрицы с большим числом обусловленности называются плохо обусловленными, в противном случае --- хорошо обусловленными.
		
		Из оценки $\|\delta x\|\leqslant\|A^{-1}\|\|\delta f\|$ следует, что чем меньше определитель $A$, тем больше определитель $A^{-1}$, а значит, больше постоянная при $|\delta f\|$ и, соответственно, больше влияния погрешностей правой части на погрешности решения.
		
		\item \textbf{Как упрощается оценка числа обусловленности, если матрица является:
		\begin{enumerate}
			\item диагональной;
			\item симметричной;
			\item ортогональной;
			\item положительно определённой;
			\item треугольной?
		\end{enumerate}}
		
		\item \textbf{Применимо ли понятие числа обусловленности к вырожденным матрицам?}
		
		Понятие числа обусловленности не применимо к вырожденным матрицам, так как они не имеют обратных. Из выражения для числа обусловленности $$\text{cond} A=\|A\|\cdot\|A^{-1}\|$$ следует, что в таком случае посчитать число обусловленности для вырожденной матрицы невозможно, и его принято считать бесконечностью.
		
		\item \textbf{В каких случаях целесообразно использовать метод Гаусса, а в каких—методы, основанные на факторизации матрицы?}
		
		Метод Гаусса эффективен, когда нужно решить систему с одной правой частью, матрица не имеет специальной структуры (не симметричная, не положительно определенная и т.д.),требуется простота реализации для разовых вычислений.
		
		Методы факторизации целесообразны, когда нужно решить много систем с одной матрицей и разными правыми частями (факторизацию делают один раз), матрица имеет специальные свойства.
		
		\item \textbf{Как можно объединить в одну процедуру прямой и обратный ход метода Гаусса? В чём достоинства и недостатки такого подхода?}
		
		Для того, чтобы объединить прямой и обратный ход метода Гаусса в одну процедуру, можно на каждом шаге прямого хода не только нормировать ведущий элемент, но и занулять остальные элементы этого столбца. Тогда в одном большом цикле мы сможем сразу получить решение системы уравнений.
		
		\textit{Достоинства:}
		\begin{itemize}
			\item решение получено сразу, без использования обратного хода;
			\item интуитивно понятный метод, связанный с элементарными преобразованиями строк матрицы.
		\end{itemize}
		
		\textit{Недостатки:}
		\begin{itemize}
			\item меньше устойчивость численного решения: быстрее накапливается ошибка из-за операций над элементами как под главной диагональю, так и над ней;
			\item большее число операций: требует примерно в 2 раза больше числа арифметических операций по сравнению с методом Гаусса
		\end{itemize}
		
		\item\textbf{Объясните, почему, говоря о векторах, норму $\|\cdot\|_1$ часто называют октаэдрической, норму $\|\cdot\|_2$~--- шаровой, а норму $\|\cdot\|_\infty$~--- кубической.}
		
		\begin{equation*}
			\|x\|_1 = \sum \limits_{i = 1}^n |x_i|, \qquad \|x\|_2 = \left( \sum \limits_{i = 1}^n x_i^2 \right)^{1/2}, \qquad \|x\|_\infty = \max_i |x_i|;
		\end{equation*}
		
		Соотвествующие единичные шары ($\|x\| \le 1$) в данных нормах при изображении в декартовых координатах представляют собой: октаэдр для $\|\cdot\|_1$, шар для~$\|\cdot\|_2$, куб для $\|\cdot\|_\infty$.
		
	\end{enumerate}
	
\end{document}