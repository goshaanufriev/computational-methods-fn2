% !TeX spellcheck = ru_RU-Russian
\documentclass[12pt, a4paper]{article}

\usepackage[utf8]{inputenc}
\usepackage[T1]{fontenc}
\usepackage[russian]{babel}

\usepackage[oglav, boldsect, eqwhole, figwhole, %
remarks, hyperref, hyperprint]{fn2kursstyle}

\makeatletter
\renewcommand{\@oddhead}{\vbox{Ануфриев Г.А., Брегадзе З.Г. \hfill ФН2-52Б \hrule}}
\makeatother

\begin{document}
	\section*{Ответы на контрольные вопросы}
	\begin{enumerate}
		\item\textbf{Почему условие $\|C\|<1$ гарантирует сходимость итерационных методов?}
		
		$\Box$Перезаписав СЛАУ таким образом: $x = Cx + y$, можно получить соотношение для итерационного процесса: $ x^{k+1} = Cx^k + y$  
		
		Вычтем и вычислим норму
		\begin{multline*}
			\|x^{k+1} - x \| = \|Cx^k + y - Cx - y\| = \|Cx^k - Cx \| \leqslant \|C\| \|x^k - x \| = \\
			= \|C\|  \|Cx^{k-1} + y - Cx - y\| \leqslant \|C\|^2 \|x^{k-1} - x \| < ... < \|C\|^{k+1} \|x^0 - x \| 
		\end{multline*}
		
		Т.к. $\|C\| \leqslant 1$, $\|C\|^{k+1}$ $\rightarrow 0 \text{, при $k \rightarrow \infty$} \Rightarrow \|C\|^{k+1} \|x^0 - x \|  \rightarrow 0 \Rightarrow \|x^{k+1} - x \| \rightarrow 0$, т.е. $x^{k+1} \rightarrow x$ $\Box$
		\item\textbf{Каким следует выбирать итерационный параметр $\tau$ в методе простой итерации для увеличения скорости сходимости? Как выбрать начальное приближение $x^0$?}
		
		Параметр $\tau$ --- число, на которое умножают исходную систему, чтобы улучшить скорость сходимости. Он подбирается так, чтобы выполнялась оценка $\|C\|<1$ и норма $C$ была как можно меньше ($C=-(A-E)$).
		
		Так происходит из-за того, что для сходимости метода необходимо, чтобы спектральный радиус матрицы системы был меньше 1. Грубо можно оценить спектральный радиус нормой матрицы, которая, в свою очередь оценивается максимальным собственным значением. Тогда получим: $\rho(A)\leq\|A\|<1$.
		
		Обычно начальное приближение $x^0$ выбирается произвольно, но есть несколько соображений, позволяющие упростить выбор:
		
		\begin{itemize}
			\item выбирается нулевое решение: $\vec{X_0}=\vec{0}$;
			\item выбирается вектор правой части, домноженный на коэффициент $\tau$: $\vec{X_0}=\tau b$.
		\end{itemize}
		
		\item\textbf{На примере системы из двух уравнений с двумя неизвестными дайте геометрическую интерпретацию метода метода Якоби, метода Зейделя, метода релаксации.}
		
		В методе Якоби итерационный процесс организован в систему:
		\begin{align}
			a_{11}x_1^{k+1}+a_{12}x_2^k=f_1;\label{I} \\
			a_{21}x_1^{k}+a_{22}x_2^{k+1}=f_2.\label{II}
		\end{align}
		
		Прямая I соответствует уравнению (\ref{I}), прямая II --- уравнению (\ref{II}). Ни одно из приближений $x^k$ не лежит на прямых I и II.
		
		В методе Зейделя итерационный процесс задается формулами:
		\begin{align}
			a_{11}x_1^{k+1}+a_{12}x_2^k&=f_1;\label{III} \\
			a_{21}x_1^{k+1}+a_{22}x_2^{k+1}&=f_2.\label{IV}
		\end{align}
		
		Решение уравнения (\ref{IV}) всегда точное.
		
		В методе релаксации каждое следующее приближение задается формулами:
		\begin{align}
			a_{11}(x_1^{k+1}-x_1^k)=\omega(-a_{11}x_1^k-a_{12}x_2^k+)
		\end{align}
				
		\item\textbf{При каких условиях сходятся метод простой итерации, метод Якоби, метод Зейделя и метод релаксации? Какую матрицу называют положительно определенной?}
		
		\textbf{Теорема}. Пусть A --- симметричная положительно определённая матрица, $\tau>0$ и выполнено неравенство $B-0.5\tau A>0$. Тогда стационарный итерационный метод $B\dfrac{x_{k+1}-x_k}{\tau}+Ax^k=f$ сходится.
		
		\textbf{Следствие 1.} Пусть $A$ --- симметричная положительно определённая матрица с диагональным преобладанием, т.е.
		$$
		a_{ii}>\sum_{j\neq i}|a_{ij}|, i = 1,2,\ldots,n.
		$$
		Тогда метод Якоби сходится.
		
		\textbf{Следствие 2.} Пусть $A$ --- положительно определенная матрица. Тогда метод релаксации сходится при $0<\omega<2$. В частности, сходится метод Зейделя ($\omega=1$).
		
		\textbf{Следствие 3.} Метод простой итерации сходится при $\tau<2/\lambda_{max}$, где $\lambda_{max}$ --- максимальное собственное значение симметричной положительно определенной матрицы $A$.
		
		Линейный оператор $A$ в действительном гильбертовом пространстве $H$ называется положительно определенным, если
		\[
		\exists\delta>0: \forall x \in H: (Ax,x)\geqslant \delta(x,x).
		\]
		
		\item\textbf{Выпишите матрицу C для методов Зейделя и релаксации.}
		
		Канонический вид метода релаксации:
		
		\[
		(D + \omega L) \frac{x^{k+1} - x^k}{\omega} + Ax^k = f, \quad k = 0, 1, 2, \ldots
		\]
		
		Преобразуем это выражение:
		
		\[
		(D + \omega L) \frac{x^{k+1} - x^k}{\omega} + Ax^k = f 
		\Leftrightarrow (D + \omega L)(x^{k+1} - x^k) = \omega(f - Ax^k) 
		\Leftrightarrow
		\]
		
		\[
		\Leftrightarrow (D + \omega L)x^{k+1} = \omega f - \omega Ax^k + (D + \omega L)x^k 
		\Leftrightarrow (D + \omega L)x^{k+1} = \omega f + x^k(-\omega A + D + \omega L)
		\]
		
		\( C = (D + \omega L)^{-1}(-\omega A + D + \omega L) \). Метод Зейделя является частным случаем метода релаксации при \(\omega = 1\), поэтому для Метод Зейделя:
		
		\[
		C = (D + L)^{-1}(-A + D + L) = (D + L)^{-1}(-U) = -(D + L)^{-1}U
		\]
		\item\textbf{Почему в общем случае для остановки итерационного процесса нельзя использовать критерий $\|x^k-x^{k-1}\|<\varepsilon$?}
		
		В общем случае последовательность $\{x_k\}_{k=1}^\infty$ может не сходится к вектору решения $x$, поэтому таким критерием пользоваться нельзя. Может быть преждевременная остановка цикла из-за малого изменения решения, но к истинному решению цикл сходиться не будет.
		
		\item\textbf{Какие еще критерии останова итерационного процесса Вы можете предложить?}
		
		Возможны следующие критерии останова:
		
		\begin{itemize}
			\item\textbf{По малости невязки}: $\|f-Ax^k\|\leq\varepsilon$
			
			Аналогично вопросу №6, полученное решение может удовлетворять неравенству, но будет находиться далеко от истинного решения.
			\item\textbf{Для методов простой итерации и Якоби:} $\|x^k-x^{k+1}\|\leq\dfrac{1-\|C\|}{\|C\|}\varepsilon$
			
			Этот критерий гарантирует точность в том случае, если $\|C\|\leq\dfrac12$. Он является точным, но требует вычисления матрицы $C$.
			\item\textbf{Для метода Зейделя:} $\|x^k-x^{k+1}\|\leq\dfrac{1-\|C\|}{\|C_U\|}\varepsilon$.
		\end{itemize}
		
		
	\end{enumerate}
\end{document}