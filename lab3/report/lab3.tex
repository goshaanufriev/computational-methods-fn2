% !TeX spellcheck = ru_RU-Russian
\documentclass[12pt, a4paper]{article}

\usepackage[utf8]{inputenc}
\usepackage[T1]{fontenc}
\usepackage[russian]{babel}
\usepackage{tabularx}

\usepackage[oglav, boldsect, eqwhole, figwhole, %
remarks, hyperref, hyperprint]{fn2kursstyle}

\makeatletter
\renewcommand{\@oddhead}{\vbox{Ануфриев Г.А., Брегадзе З.Г. \hfill ФН2-52Б \hrule}}
\makeatother

\begin{document}
	\section*{Ответы на контрольные вопросы}
	\begin{enumerate}
		\item\textbf{Определите количество арифметических операций, требуемое для интерполирования функции в некоторой точке многочленом Лагранжа (включая построение самого многочлена) на сетке с числом узлов, равным n.}
		
		Пусть задана сетка, узлы которой $x_1, x_2, \ldots, x_n$, и интерполируется функция $f(x)$. Тогда полином Лагранжа будет иметь вид:
		$$
		L_n(x)=\sum_{i=1}^n f(x_i) \prod_{\substack{j=1, \\ i \neq j}}^{n}\left(\frac{x-x_j}{x_i-x_j}\right).
		$$
		
		В таком случае будут иметь место:
		\begin{itemize}
			\item по $(n-1)$ операции умножения в числителе и знаменателе дроби, 1 операция деления и 1 операция умножения результата на $f(x_i)$;
			\item $2(n-1)$ операций вычитания во время нахождения коэффициентов $c_k$
			\item каждые из выше перечисленных операций повторяются $n$ раз
			\item также есть $n-1$ операция сложения чисел $f(x_i)\cdot c_i(x_i)$
		\end{itemize}
		Итого интерполяция многочленом Лагранжа потребует $(n-1+n-1+1+1)\cdot n + 2(n-1)\cdot n + n-1=2n^2+2n^2-2n+n-1=4n^2-n-1\sim O(4n^2),\: n\to\infty$.
		
		\item \textbf{Определите количество арифметических операций, требуемое для интерполирования функции в некоторой точке кубическим сплайном (включая затраты на вычисление коэффициентов сплайна) на сетке с числом узлов, равным $n$.}
		
		\item \textbf{Функция $f(x)=e^x$ интерполируется многочленом Лагранжа на отрезке $\left[0,2\right]$ на равномерной сетке с шагом $h=0{,}2$. Оцените ошибку экстраполяции в точке $x=2{,}2$, построив многочлен Лагранжа и подставив в него это значение, а также по формуле для погрешности экстраполяции.}
		
		Построенный полином Лагранжа в точке $x=2{,}2$ отличается от исходной функции на величину:
		$$\Delta=6{,}26641\cdot10^{-8}.$$
		
		Погрешность экстраполяции при $x\in\left[2, 2{,}2\right]$ будет определяться по формуле
		$$
		|y(x)-L_n(x)|\leq h^{n+1}\cdot \max_{\xi\in\left[0,\: 2{,}2\right]}|y^{(n+1)}(\xi)|
		$$
		
		В данном случае $n=10,\:h=0{,}2$. Любой производной $y(x)=e^x$ является функция $f(x)=e^x$, она монотонно возрастающая, поэтому своё максимальное значение на отрезке $\left[0,\: 2{,}2\right]$ будет принимать в точке $x=2{,}2$. Тогда
		$$
		|y(x)-Ln(x)|\leq0{,}2^{11}\cdot e^{2{,}2}\approx1{,}84832\cdot10^{-7}.
		$$
		
		Таким образом, ошибка экстраполяции с помощью полинома Лагранжа меньше теоретических предположений.
		
		\item \textbf{Выпишите уравнения для параметров кубического сплайна, если в узлах $x_0$ и $x_n$ помимо значений функции $y_0$ и $y_n$ заданы первые производные $y'(x_0)$ и $y'(x_n)$.}
		
		\item\textbf{Каковы достоинства и недостатки сплайн-интерполяции и интерполяции многочленом Лагранжа?}
		
		\begin{table}[ht]
			\centering
			\begin{tabularx}{\textwidth}{|X|X|X|}
				\hline
				& Достоинства & Недостатки \\
				\hline
				Многочлен Лагранжа & 
				\begin{itemize}
					\item простой способ нахождения полинома Лагранжа
					\item наличие непрерывных производных больших порядков
					\item интерполяционный полином Лагранжа задается единой на всем отрезке формулой
				\end{itemize}
				 & 
				 \begin{itemize}
				 	\item чем больше узлов сетки --- тем сложнее построить интерполяционный полином Лагранжа и вычислить значение интерполянта в произвольной точке
				 	\item сильная зависимость точности интерполянта от вида сетки (на чебышевской может хорошо приближать функцию, а на равномерной, при том же количестве узлов, могут происходить осцилляции)
				 \end{itemize}
				 \\
				\hline
				Сплайн-интерполяция &
				\begin{itemize}
					\item требуется малое количество операций для вычисления коэффициентов $a_i,\,b_i\,c_i\,d_i$ многочлена $S_3(x)$
					\item нахождение значения функции $S_3(x)$ в произвольной точке требует небольшое число арифметических операций
					\item если $f(x)\in C^4\left[a,\,b\right]$, то с помощью сплайна $S(x)$ можно приблизить не только  функцию, но и её первую и вторую производные.
				\end{itemize}
				&
				\begin{itemize}
					\item
				\end{itemize}
				\\
				\hline
			\end{tabularx}
		\end{table}
	\end{enumerate}
\end{document}